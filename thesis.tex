\documentclass[a4paper, 12pt]{article}
% Nyelvi beállítások
%\def\magyarOptions{defaults=hu-min}
\usepackage[english]{babel}
\usepackage[utf8]{inputenc}
\usepackage{amsmath}
\usepackage{amsfonts}
\usepackage{amssymb}
\usepackage{amsthm}
\usepackage{graphics}
\usepackage{tikz}
\usepackage{t1enc}
\usepackage{mathrsfs}
\usepackage{mathptm}
\usepackage{physics}
\usepackage{times}
%\usepackage{euler}
%\usepackage{concrete}
%\usepackage{newtxtext}
%\usepackage{newtxmath}
\usepackage{appendix}
\usepackage[nottoc,numbib]{tocbibind}
\usepackage{geometry}
 
% Sorköz és teljes oldalas kitöltés
\linespread{1.3}
\usepackage{fullpage}

% Hiperlinkek
\usepackage[pdfauthor={},pdftitle={}]{hyperref}
\hypersetup{colorlinks=true, linkcolor=blue, citecolor=red, filecolor=magenta, 
            urlcolor=cyan, linktocpage=true}

% Sorszámozás            
%\renewcommand{\thesection}{\thechapter.\arabic{section}}
%\numberwithin{equation}{section}


% Tételek, definíciók
%\swapnumbers
\newtheorem{lem}{Lemma}[section]
\newtheorem{theo}[lem]{Theorem}
\newtheorem{state}[lem]{Proposition}
\newtheorem{defin}[lem]{Definition}
\newtheorem{note}[lem]{Note}
\newtheorem{example}[lem]{Example}
\newtheorem{corollary}[lem]{Corollary}
\newtheorem{remark}[lem]{Remark}

% Megjegyzésekhez, amik nem szerepelnek majd végül a pdf-ben
\newcommand{\ignore}[1]{}

% Speciális jelölések
\newcommand{\simp}{\mathop{\mathrm{Simp}}}
\newcommand{\graph}{\mathop{\mathrm{graph}}}
\newcommand{\dom}{\mathop{\mathrm{dom}}}

% Tartalomjegyzék mélysége
\setcounter{tocdepth}{4}
%\renewcommand{\bibliofont}{\normalsize}
\addto\captionsenglish{\renewcommand{\refname}{Bibliography}}

% másik qed: ezt használom azokra a tételekre, amiket nem fogok bizonyítani
\newcommand*{\qedb}{\hfill\ensuremath{\blacksquare}}

% Irodalomjegyzék
\usepackage[backend=bibtex,
backref=true,
style=alphabetic,
citestyle=alphabetic]{biblatex}
\addbibresource{references}

\begin{document}
\thispagestyle{empty}
\newgeometry{
    top=2.5cm,
    bottom=2.5cm,
    outer=2.5cm,
    inner=2.5cm,
}
% Kezdőlap
\begin{center}\renewcommand\baselinestretch{0.9}
{\Large \textsc{KTH Royal Institute of Technology}\\} \hrulefill
\vspace{1.0cm} {\huge \\ Benjámin Martin Seregi \\} \vspace{1cm}
{\Huge\textsc{On the list coloring of $k$-band buffering cellular graphs}\\ \vspace{0.5cm}}
{\large\textsc MSc Thesis \\ Degree Project in Electrical Engineering}\\ \vspace{1cm}
{\large \textit{Examiner}\\
\vspace{0.2cm}
Marina Petrova\vspace{1.2cm}}\\
{\large \textit{Supervisors}\\
\vspace{0.2cm}
Dávid Kunszenti-Kovács and Göran Andersson\vspace{1.2cm}}\\
\begin{figure}[!h]
\begin{center}
\resizebox{5.5cm}{!}
{\includegraphics{KTH_Logotyp_RGB_2013}}
\end{center}
\end{figure}
{\large  Information and Communication Technology \\ \vspace{0.5cm}
Stockholm, Sweden\\2018}
\end{center}

\pagebreak
\pagenumbering{Roman}
\section*{Acknowledgements}
%\vspace*{\fill} 
First of all, I would like to express my deep gratitude to Dávid Kunszenti-Kovács, my thesis supervisor, for his patient guidance.

I would also like to thank my parents for their continuous support throughout my studies.
%\vspace*{\fill}
\newpage
\tableofcontents
\newpage
\pagenumbering{arabic}
\section{Introduction}
In telecommunication one of the most challenging problems is the efficient allocation of available frequency. When the available bandwidth is limited, the efficient utilization of the frequency spectrum is a major concern. Due to the growing number of mobile Internet users, optimal channel allocation in cellular networks and their variants have been heavily researched in recent years \cite{Audhya:2011:SCA:1988563.1988571}.

Several variants of the channel allocation problem have been defined based on the different channel constraints that a particular service might require. One of them is the so-called co-channel constraint where the same channel is not allowed to be assigned to neighboring cells simultaneously. This problem has been formalized as a graph coloring problem by many authors \cite{1456167}. Unfortunately, graph coloring is a well-known NP-complete problem \cite{Kar72} and therefore we do not know if a polynomial time algorithm for co-channel constraint satisfaction exists. Therefore various heuristic algorithms have been developed, the list of methods includes genetic algorithms, neural networks, graph-based, and other approaches \cite{Audhya:2011:SCA:1988563.1988571}.

Cellular network topologies are usually idealized as a certain geometric structure. The most common network structure is the hexagonal grid topology where each cell is represented by a regular hexagon (two cells are neighbors if they share a common boundary). In \cite{662943}, Sen, Roxborough, and Medidi exploited this special structure and proposed an algorithm that optimally solves the channel allocation problem in $k$-band buffering systems where $k$ is $1$ or $2$. Moreover, the algorithm has polynomial running time $O(p)$ where $p$ is the number of cells.

R. Wang, et al. \cite{7248845} introduced a \textit{distinctly different} channel allocation problem from all the above-mentioned problems, by assuming a $2$-band buffering hexagonal cell topology (the interference graph created from this topology is called a cellular graph) where each cell has a fixed number of frequency channels (channels are either busy or free). They asked the following question: "\textit{What is smallest size of the set of free channels associated with the cells (nodes of the cellular graph) that can guarantee interference free channel assignment to all the nodes?}". This problem is related to one of the generalizations of the graph coloring problem, called list coloring. It turned out that the required number of free channels is between $8$ and $10$. In addition, two algorithms have been proposed to create an interference free assignment, that is, a list-coloring of the cellular graph. The first one is an integer linear programming formulation of the list coloring problem (and therefore it is not a polynomial algorithm), while the second one is a heuristic linear time algorithm that is, according to their experiment, within 12\% of the optimal solutions.
\newpage
\section{Section} 

\newpage
\printbibliography
\addcontentsline{toc}{section}{References}
\end{document}
